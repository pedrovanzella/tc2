\chapter{\label{chap:ai}Algoritmos de Agendamento para Elevadores}

A busca pela solução do problema de atribuir elevadores para atender chamadas
feitas pelos passageiros, minimizando alguma métrica, é apresentada pela
literatura pesquisada na forma de algoritmos~\cite{KOEHLEROTTIGER02}.
Tais algoritmos possuem complexidades distintas, indo desde algoritmos
triviais~-~mas ainda interessantes para fins de comparação~-, até soluções mais
complexas onde mais dados são utilizados de modo a tomar decisões mais
complexas.

Em um hipotético cenário ideal, ter-se-iam todos os dados de cada
passageiro~-~\textit{i.e.} cada pessoa que chegasse ao andar informaria de
antemão para qual andar deseja ir antes mesmo de entrar no elevador. No entanto,
isto não é realista no contexto dos sistemas de elevadores instalados
atualmente, onde cada pessoa apenas informa se deseja subir ou descer. Portanto,
os algoritmos aqui descritos tentam fazer inferências\footnote{\textit{e.g.} A
  lotação do elevador pode ser estimada com base no peso reportado pela balança
  interna do elevador, que já se encontra nele por motivos de segurança, ou
  abstrair a capacidade e lotação em número de pessoas.} a respeito de dados que
não possuem, quando relevante, ou tentam tomar decisões ignorando os dados que
não estão disponíveis.

A seguir são descritos, do mais simples ao mais complexo, os algoritmos
selecionados para o desenvolvimento.

\section{\label{sec:ai:nn}Nearest neighbour}

O algoritmo de \textit{Nearest Neighbour} é o mais ingênuo de todos, e servirá
de base para a avaliação dos demais algoritmos. Seu funcionamento é trivial: o
elevador mais próximo da chamada sempre atenderá esta
chamada~\cite{Friese20061908}. Um dos problemas deste algoritmo é que ele pode
causar muitas mudanças de direção de um elevador, o que acarreta um tempo de
espera maior para os passageiros que estão dentro dele.

Considere, por exemplo, o cenário da Figura~\ref{fig:elevadores:nn:bad}. Este
cenário é composto por um prédio de 8 andares com 3 elevadores. A situação dos
elevadores é a seguinte:

\begin{itemize}\setlength\itemsep{0mm}
\item $E1$ no sexto andar, com ocupação\footnote{A ocupação do elevador é representada pelo percentual dentro do círculo.} de 20\% e como destino\footnote{O destino do elevador é representado pela seta.} o segundo andar;
\item $E2$ no primeiro andar, com ocupação 10\% e como destino o sexto andar;
\item $E3$ no quarto andar, com ocupação 90\% e como destino o sétimo andar.
\end{itemize}

Neste instante, uma nova chamada de corredor é originada no sétimo andar.

\begin{figure}[htb!]
  \centering
  \includegraphics[scale=0.6]{img/elevator_example_nn_bad.eps}
  \caption{Cenário exemplo \#1 de prédio com 8 andares e 3 elevadores.}
\label{fig:elevadores:nn:bad}
\end{figure}

Caso o algoritmo de \textit{Nearest Neighbour} seja utilizado, o elevador $E1$
seria selecionado para atender a chamada no sétimo andar. Isto é claramente ruim
para os passageiros deste elevador. Além disto, é possível notar que o elevador
$E3$ já tinha uma parada programada no sétimo andar e seria possível atender a
chamada sem alterar a agenda de nenhum elevador. O algoritmo de \textit{Nearest
Neighbour}, no entanto, não leva em consideração estas informações.

O único propósito deste algoritmo é servir de base de comparação com outros
algoritmos propostos, de modo a validarmos o simulador. Espera-se que uma
melhora clara de desempenho seja notada ao comparar-se este com o próximo
trivial, o \textit{Nearest Neighbour Melhorado}.

\section{\label{sec:ai:nnm}Nearest neighbour melhorado}
Uma melhoria que pode ser feita ao algoritmo de \textit{Nearest Neighbour} é
considerar o sentido em que o elevador está indo para atender a
chamada~\cite{Friese20061908}. Isto implica em considerar-se agora a informação
de sentido das chamadas. É importante notar que ainda não se considera quantas
pessoas fizeram uma chamada~-~apenas sabe-se que há chamadas no andar, e como
destinos tem-se ``subir'', ``descer'' ou ``ambos''.

Este algoritmo resolve o problema de mudanças de direção que o algoritmo de
\textit{Nearest Neighbour} sofre.

Considere, novamente, o caso ilustrado pela Figura~\ref{fig:elevadores:nn:bad}.
Este algoritmo considerará apenas os elevadores que estão parados ou indo no
sentido de onde a chamada foi originada. Neste caso, apenas os elevadores $E2$ e
$E3$ serão considerados. O elevador $E3$ está mais próximo da chamada, então
será escolhido.

\section{\label{sec:ai:minimize-cost-function}Simple Scheduler}

Podemos definir estes algoritmos, de forma mais genérica, como funções de custo.
Estas funções são inerentes a cada elevador, descrevendo quão custoso é
atender uma chamada~\cite{Friese20061908}. A decisão de qual elevador é
escolhido para atender a chamada é feita com base em qual deles terá o menor
valor da função de custo.


Um exemplo de função de custo é:

\[
  J(e, l, p) = \lambda l(p - e)
\]

Onde:
\begin{description}[leftmargin=!,labelwidth=\widthof{\bfseries Pu}]\setlength\itemsep{0mm}
\item[$\boldsymbol{e}$] Número do andar onde o elevador se encontra
\item[$\boldsymbol{l}$] Percentual de ocupação do elevador
\item[$\boldsymbol{p}$] Número do andar onde a chamada foi originada
\item[$\boldsymbol{\lambda}$] Fator de multiplicação, que assume os seguintes valores:
  \begin{itemize}\setlength\itemsep{0mm}
    \item $0$, caso o programa atual do elevador faça com que ele passe por
      aquele andar;
    \item $1$, caso o elevador não tenha um programa (\textit{i.e.},
      ele esteja ocioso) ou o elevador esteja indo na direção da chamada;
    \item $2$, caso ele mude de direção para atender esta chamada\footnote{Multiplica-se
        a distância por dois pois é necessário ir até o andar da chamada e então
        voltar para o andar onde se estava anteriormente, para só então atender
        a chamada.}.
  \end{itemize}
\end{description}

\begin{figure}[htb!]
  \centering
  \includegraphics[scale=0.6]{img/elevator_example1.eps}
  \caption{Cenário exemplo \#2 de prédio com 8 andares e 3 elevadores.}
  \label{fig:elevadores-1}
\end{figure}

Utilizando como exemplo o cenário da Figura~\ref{fig:elevadores-1}, onde uma
chamada de corredor para descer\footnote{O sentido da chamada é representado
pelo círculo com um \textbf{D}, de \textit{Down}.} é originada no oitavo andar.
A situação de cada elevador é a seguinte:

\begin{itemize}\setlength\itemsep{0mm}
\item $E1$ no sétimo andar, com ocupação de 20\% e como destino o segundo andar;
\item $E2$ no primeiro andar, com ocupação 10\% e como destino o sexto andar;
\item $E3$ no sétimo andar, com ocupação 90\% e como destino o primeiro andar.
\end{itemize}

Podemos calcular o custo de atender a esta chamada para cada elevador do prédio.
Para o Elevador $E1$:

\[J(7, 0.2, 8) = 2 \times 0.2 \times (8 - 7) = 0.4\]

Neste caso, $\boldsymbol{\lambda}$ é 2, pois o elevador deverá mudar de sentido
(\textit{i.e.}, ele deverá subir até o oitavo andar e então voltar para o sétimo
andar, o que representa um deslocamento de dois andares).

Para o Elevador $E2$:

\[J(1, 0.1, 8) = 1 \times 0.1 \times (8 - 1) = 0.7\]

Neste caso $\boldsymbol{l}$ é $0.1$, pois sua lotação é 10\%\footnote{Como não
há informações a respeito do futuro, não é possível considerar alterações na
carga de um elevador.}, e $(p - e)$ é $7$, pois o elevador deverá subir sete
andares\footnote{Pela regra de cálculo, foi considerada apenas a posição atual
do elevador, e não a posição que ele estará ao fim de sua última atividade.
Seria possível alterar esta regra, gerando uma função de custo levemente
diferente, com resultados diferentes. É um teste válido para a implementação.}.

Para o Elevador $E3$:

\[J(7, 0.9, 8) = 2 \times 0.9\times (8 - 7) = 1.8\]

Neste caso $\boldsymbol{l}$ é $0.9$, pois sua lotação é 90\%, e
$\boldsymbol{\lambda}$ é $2$, pois o elevador $E3$ deve subir do sétimo para o
oitavo andar e descer novamente até o sétimo.

Observa-se que, para esta função de custo, neste sistema, é vantajoso mudar o
sentido de $E1$ para atender a chamada no oitavo andar, e depois continuar na
direção original. Várias funções de custo podem ser experimentadas e comparadas:
outras funções de custo levariam em consideração mudanças de direção de
viagem\footnote{\textit{E.g.}, pode ser vantajoso um elevador mudar de direção
para atender uma chamada a um andar de distância, caso a alternativa seja fazer a
chamada esperar um deslocamento de dezenas de andares de outro elevador.}, ou
ainda tentar manter todos os custos o mais baixo possível, ao mesmo tempo que o
mais próximos uns dos outros. Por exemplo, uma outra função de custo possível
seria:

\[J(e, l, p) = l(\lambda(p - e))^{2}\]

Esta função penaliza mais o movimento, elevando-o ao quadrado. Para o exemplo da
Figura~\ref{fig:elevadores-1}, tem-se:

\[J(7, 0.2, 8) = 0.2 \times (2 \times (8-7))^2 = 0.8\]
\[J(1, 0.1, 8) = 0.1 \times (1 \times (8-1))^2 = 6.4\]
\[J(7, 0.9, 8) = 0.9 \times (2 \times (8-7))^2 = 3.6\]

Novamente, para esta função, o elevador $E1$ é eleito para atender a chamada.

\subsection{Algoritmo}

\begin{algorithm}[htb!]
  \centering
    \begin{minted}[linenos,fontsize=\small]{c++}
ChooseElevator(client, elevatorSet, costFunction) {
  selectedElevador = elevatorSet.front();
  bestCost = infinite;
  for each (elevador in elevatorSet) {
    cost = costFunction(client, elevatorSet)
    if cost < bestCost {
      selectedElevator = elevator;
      bestCost = cost;
    }
  }
  return selectedElevator;
}
    \end{minted}
  \caption{\label{alg:cost-function}Minimização da \textit{função de custo}.}
\end{algorithm}


O Algoritmo~\ref{alg:cost-function} ilustra um algoritmo com o comportamento desta
função. Os argumentos para este algoritmo são:

\begin{description}[leftmargin=!,labelwidth=\widthof{\bfseries $costFunction$}]\setlength\itemsep{0mm}
  \item[$client$] Abstração de um cliente;
  \item[$elevatorset$] Conjunto de todos os elevadores do prédio;
  \item[$costFunction$] Ponteiro para a função de cálculo de custo.
\end{description}

Seu funcionamento pode ser descrito como: dada uma chamada, para cada elevador
presente no conjunto calcula-se a função de custo e o algoritmo retorna o
elevador com a menor função de custo.

\subsection{Nearest neighbour como função de custo}

É possível definir o algoritmo de \textit{Nearest Neighbour}
(Seção~\ref{sec:ai:nn}) como uma função de custo:

\[J(e, p) = p - e\]

Onde considera-se apenas $p - e$, que é a distância entre a posição atual do
elevador ($e$) e o número do andar onde a chamada foi originada ($p$).

\section{\label{sec:ai:planning}Planning}

Neste algoritmo é realizada a expansão no espaço de estados em uma árvore de
decisões, avaliando-se para cada passo o tempo que um elevador levará para
chegar até o estado desejado, vários passos no
futuro~\cite{Koehler00elevatorcontrol}, como em um jogo de xadrez. Em cada nodo
desta árvore de decisões (Figura~\ref{fig:planning}) é feita a pergunta:
\textit{qual elevador deve atender qual chamada}?

A expansão da árvore pode ocorrer diversas vezes, fazendo com que a árvore de
decisões possua múltiplos níveis. No fim desta expansão, que se dá quando
o horizonte de expansão é atingido, é realizado o somatório, a partir da raiz até
cada folha da árvore, dos custos calculados. No fim destes somatórios, o
algoritmo toma a decisão pelo ramo com menor custo (a partir da raiz). Assim,
avança-se um passo na simulação e executa-se o algoritmo novamente.

O parâmetro que limita a expansão da árvore é chamado de \textit{horizonte de
cálculo}. O valor deste horizonte é arbitrário e representa o número de níveis
da expansão. Em uma regra geral, quanto maior o horizonte de cálculo maior a
chance do algoritmo tomar a melhor escolha. Porém, a complexidade de memória
$O(N^{N})$, onde $N$ é o horizonte de cálculo e $E$ é o número de elevadores,
é um fator limitador do horizonte de cálculo.

Para cada nodo da árvore, instancia-se um simulador novo e executa-se esta
simulação até o momento em que o elevador atenderia o chamado do cliente. Isto é
importante para ter-se uma noção realista do comportamento dos outros
elevadores. A única diferença deste simulador instanciado para um simulador real
está na geração de novos eventos. Estes novos simuladores não geram eventos de
chegada de cliente, dado que é impossível, no mundo real, prever quando um
cliente novo chegará.\footnote{O simulador tem esta capacidade, mas isto não
  seria justo, já que um sistema real não tem como ter este tipo de informação.}

Na Figura~\ref{fig:planning}, vemos um exemplo hipotético de \textit{planning}
sendo executado com $N = 3$ e $E = 2$. Para cada nível da árvore a decisão a ser
tomada é ``qual elevador deve atender a próxima chamada da fila?'', e, para
isto, calcula-se o custo de tempo para cada elevador. O resultado do custo
pode ser visto entre parênteses em cada nodo da árvore da
Figura~\ref{fig:planning}.

Ao atingir o horizonte (no caso da Figura~\ref{fig:planning}, no nível 3 da
árvore), soma-se os custos até cada folha (na Figura~\ref{fig:planning}, os
círculos abaixo do nível mais baixo representam a soma dos custos até aquela
folha). O caminho de menor custo total é o que deve ser tomado. No
exemplo da Figura~\ref{fig:planning}, o elevador $E2$ será agendado para atender
a primeira chamada da fila. O agendamento das demais chamadas é gerado somente
na próxima execução do \textit{scheduler}. Isto se dá pelo fato de que o estado
do simulador pode ter sido alterado pela chegada de um cliente neste meio tempo.
Mais detalhes podem ser vistos no Capítulo~\ref{chap:model}.

\begin{figure}[htb!]
  \centering
  \includegraphics[scale=0.6]{img/planning.eps}
  \caption{Exemplo de \textit{planning} com horizonte 3 e dois elevadores.}
\label{fig:planning}
\end{figure}