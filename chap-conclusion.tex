\chapter{\label{chap:conclusion}Conclusão}

Após a realização deste estudo, foi possível enxergar claramente a simulação
como uma excelente ferramenta para estudar e experimentar sistemas. No objeto
deste estudo, os resultados obtidos através das simulações evidenciam que é
possível diminuir o tempo médio de espera de sistemas de elevadores substituindo
o software do sistema de controle dos mesmos por alternativas que utilizem
técnicas algorítmicas mais avançadas em relação às utilizadas pela indústria.
Deste modo, é possível beneficiar não só usuários de futuras instalações, mas
também usuários de sistemas já existentes.

Neste contexto, a escolha da uma estratégia de agendamento adequada possui alta
relevância. Conforme as estatísticas apresentadas no
Capítulo~\ref{chap:results}, a utilização de uma estratégia inadequada para
determinado cenário pode causar um aumento significativo no tempo de espera
médio. Pode-se verificar a validade desta afirmação ao notar que, nos cenários
\textit{low-rise}, \textit{high-rise} e \textit{skyscraper}, a melhor estratégia
foi 41.31\%, 66.57\% e 82.27\% superior à pior opção, respectivamente.

Dentre as estratégias construídas durante este trabalho, o \textit{Planning}
obteve destaque positivo nos resultados. Ao avaliar esta estratégia obteve-se o
menor tempo de espera médio em todos os cenários. Os resultados foram 5.41\%,
35.74\% e 72.71\% superiores em relação à segunda melhor estratégia nos cenários
\textit{low-rise}, \textit{high-rise} e \textit{skyscraper}, respectivamente.
Embora estes números sejam positivos, é preciso observar que cenário com menos
andares (\textit{low-rise}), os benefícios de tal estratégia foram menos
expressivos. Porém, o ganho de desempenho cresce à medida que o prédio cresce.
Em fase disto, pode ser que os a implantação desta solução em prédios menores
não compense o benefício de sua utilização - embora esta análise esteja fora do
escopo deste trabalho. Após o \textit{Planning}, a estratégia que obteve
melhores resultados foi o \textit{Better Nearest Neighbour} - uma opção de
implementação trivial e baixo custo computacional que pode ser uma opção
interessante para prédios menores. Isto mostra que estratégias triviais
possivelmente já sejam suficientes para prédios pequenos, enquanto estratégias
mais complexas tornam-se mais atraentes à medida que os prédios e o número de
elevadores aumentam

Devido ao cronograma enxuto para a elaboração deste estudo, não houve tempo
hábil para expandir as possibilidades de algoritmos aplicáveis a este problema.
De fato, a maior parcela do tempo destinado a este projeto foi consumida pelo
projeto, implementação e validação do simulador. Isto se justifica pelo fato de,
para gerar resultados confiáveis e dignos de comparação entre si, o simulador e
o sua modelagem devem estar corretas. Do contrário, os dados gerados pela
simulação estariam postos em xeque.

Sendo assim, fica em aberto a questão: podemos melhorar ainda mais? Este
questionamento faz valer a pena o tempo e esforço dedicados a criação do
simulador. Pois, ao mesmo tempo em que serve como ferramenta para validação das
estratégias propostas e implementadas, este também serve como uma plataforma
flexível e expansível para construção e validação de novas estratégias no
futuro.

\section{Trabalhos Futuros}

A literatura sugere que algumas técnicas não são vantajosas. Por exemplo, o uso
de algoritmos genéticos~\cite{KOEHLEROTTIGER02} para a definição de zonas de
atuação não é uma boa solução, bem como outras que forçam o usuário a descobrir
qual carro atenderá seu chamado~\cite{KOEHLEROTTIGER02}. O mesmo artigo mostra que o sistema de controle
de destino~\cite{KOEHLEROTTIGER02} obtém bons resultados, mas também sofre do
problema de onerar o usuário.

Já os artigos vistos trouxeram algumas das soluções mais promissoras: o
primeiro, a utilização de lógica~\textit{fuzzy} para reconhecimento de padrões
de tráfego e ajuste dinâmico do comportamento dos carros~\cite{marja97}; o
segundo, propondo um modelo estatístico~\cite{DBLP:journals/corr/abs-1212-2499}
que resultou em uma métrica para sucesso: o tempo de espera do usuário sendo
reduzido de 5\% a 55\% em comparação com o algoritmo
trivial~\cite{DBLP:journals/corr/abs-1212-2499}.

Trabalhos futuros podem analisar estas sugestões, utilizando-se do
\textit{framework} construído neste trabalho. Outras políticas mais simples
também podem ser avaliadas, implementando novas funções de custo que levam em
consideração mais dados.

Há também uma gama de políticas de ociosidade dos
elevadores\footnote{\textit{i.e.}, o que fazer quando o elevador está
  parado~-~deixá-lo no mesmo lugar, ou levá-lo para algum outro? Como tomar esta
decisão?}, que pode ser testada em combinação com as políticas de escolha de elevador.

Simulações mais ricas podem ser obtidas variando as distribuições de
probabilidade de chegada e saída nos andares. Por exemplo, variando os
parâmetros com o tempo, para fazer mais clientes chegarem no lobby pela manhã ou
simular um prédio onde há muito tráfego entre um par de andares durante algumas
horas do dia.

Trabalhos mais simples, mas não por isso menos interessantes, podem avaliar o
impacto da escolha do número de elevadores para um determinado prédio, ou ainda
o impacto de políticas de exclusão de atendimento por alguns elevadores a alguns
andares~-~\textit{i.e.} um elevador que só atende certos andares.

Há, sem dúvida, muitas opções para trabalhos futuros que utilizem o conhecimento
construído neste trabalho, e que poderão gerar valor real, tanto acadêmico
quanto para a indústria.
