\chapter{\label{chap:conclusion}Conclusão}

Sendo os elevadores um meio de transporte presente na vida cotidiana de tantos
habitantes de grandes centros urbanos, eles são um alvo para otimização. Porém,
ao contrário do que acreditava-se antes da realização deste trabalho, encontrar
uma solução para este problema está longe do trivial. Nas últimas décadas,
diversas linhas de pesquisa científica~-~dentro e fora da indústria e de seus
segredos industriais e patentes~-~buscaram evoluir os sistemas de elevadores e a
forma com que seus usuários os utilizam.

Um simulador que permita a avaliação objetiva e comparação de diferentes
estratégias em diferentes cenários foi de grande valor, e espera-se que continue
se mostrando de valor também para pesquisas futuras e para o próprio mercado de
fabricantes de elevadores. Isto por que o simulador é uma plataforma de
validação de estratégias~-~e não só as estratégias implementadas durante o tempo
limitado dedicado para este estudo mas também como qualquer estratégia que venha
a ser implementada no futuro.

Notou-se que estratégias que consideram mais dados~-~como a ocupação do
elevador~-~muitas vezes geram resultados melhores com um baixo custo
computacional. Ainda que o ganho seja pequeno, o custo de implementação é,
muitas vezes, essencialmente nulo.

A estratégia de \textit{Planning} se mostrou muito vantajosa, principalmente
conforme o tamanho do prédio aumenta. No entanto, o processamento necessário
para realizá-la é bastante grande, e o custo de implementação em sistemas reais
pode se provar proibitivo. Além disto, o custo computacional cresce bastante
conforme o número de andares e de elevadores aumenta. Horizontes maiores que $2$
se mostraram proibitivamente caros, em tempo de execução, para prédios com mais
de $100$ andares. Mesmo para prédios menores, com até $20$ andares, horizontes
maiores que $5$ não são viáveis. No entanto, isto não é um problema tão grande,
pois
\unsure{Law of diminishing returns. Como exmplico isso? Cada vez ganhamos
  ``menos a mais''. (Vanzella)}

\section{Trabalhos Futuros}

A literatura sugere que algumas técnicas não são vantajosas. Por exemplo, o uso
de algoritmos genéticos~\cite{KOEHLEROTTIGER02} para a definição de zonas de
atuação não é uma boa solução, bem como todas as outras que forçam o usuário a
descobrir qual carro atenderá seu chamado. O mesmo artigo mostra que o sistema
de controle de destino~\cite{KOEHLEROTTIGER02} obtém bons resultados, mas também
sofre do problema de onerar o usuário.

Já os artigos vistos trouxeram algumas das soluções mais promissoras: o
primeiro, a utilização de lógica~\textit{fuzzy} para reconhecimento de padrões
de tráfego e ajuste dinâmico do comportamento dos carros~\cite{marja97}; o
segundo, propondo um modelo estatístico~\cite{DBLP:journals/corr/abs-1212-2499}
que resultou em uma métrica para sucesso: o tempo de espera do usuário sendo
reduzido de 5\% a 55\% em comparação com o algoritmo
trivial~\cite{DBLP:journals/corr/abs-1212-2499}.

Trabalhos futuros podem analizar estas sugestões, utilizando-se do
\textit{framework} construído neste trabalho. Outras políticas mais simples
também podem ser avaliadas, implementando novas funções de custo que levam em
consideração mais dados.

Há também uma gama de políticas de ociosidade dos
elevadores\footnote{\textit{i.e.}, o que fazer quando o elevador está
  parado~-~deixá-lo no mesmo lugar, ou levá-lo para algum outro? Como tomar esta
decisão?}, que pode ser testada em combinação com as políticas de escolha de elevador.

Simulações mais ricas podem ser obtidas variando as distribuições de
probabilidade de chegada e saída nos andares. Por exemplo, variando os
parâmetros com o tempo, para fazer mais clientes chegarem no lobby pela manhã ou
simular um prédio onde há muito tráfego entre um par de andares durante algumas
horas do dia.

Trabalhos mais simples, mas não por isso menos interessantes, podem avaliar o
impacto da escolha do número de elevadores para um determinado prédio, ou ainda
o impacto de políticas de exclusão de atendimento por alguns elevadores a alguns
andares~-~\textit{i.e.} um elevador que só atende certos andares.

Há, sem dúvida, muitas opções para trabalhos futuros que utilizem o conhecimento
construído neste trabalho, e que poderão gerar valor real, tanto acadêmico
quanto para a indústria.
