\chapter{\label{chap:objectives}Objetivo do Trabalho}

O objetivo deste trabalho é comparar, através de simulações, diferentes
estratégias de controle de elevadores utilizando Inteligência Artificial em
cenários distintos. Os resultados das simulações são avaliados e, dentre as
opções possíveis, são verificadas quais estratégias resultam em melhor
desempenho no transporte de passageiros para cada cenário. Tal melhora se
reflete em duas métricas principais: \textit{Waiting Time} e \textit{Journey
Time}. Entretanto, a meta principal é a redução do \textit{Average Waiting
Time}, ou tempo médio de espera. A preferência por esta métrica, em detrimento
do \textit{Average Journey Time}, ou tempo médio de jornada, se dá pois, uma vez
dentro do elevador, o passageiro não se sente mais esperando: ele sente que já
está sendo servido, conforme constatado na seção \ref{section:data}.

O simulador carrega uma lista de cenários a partir de um arquivo de configuração
e realiza a simulação de cada cenário. É possível comparar várias estratégias em
um mesmo cenário. Para isto, foram implementados dois algoritmos de agendamento
- \textit{simple} e \textit{planning}, detalhados na seção XXX - e 4 funções de
custo - \textit{random}, \textit{nearest neighbour}, \textit{better nearest
neighbour} e \textit{weighted}, detalhados na seção XXX. Os algoritmos e as
funções de custo podem ser utilizados em uma combinação entre si.

\unsure{Pedro, colocar referências corretas para as seções.}

Após as simulações, o simulador fornece um relatório apresentando
métricas\footnote{Seção~\ref{section:data}.} de desempenho para sistemas de
controle de grupo de elevadores. Além do relatório, o simulador é capaz de
gerar, a partir das métricas obtidas, os seguintes gráficos: \textit{Clientes
por Elevador}, \textit{Chegadas por Andar}, \textit{Desembarques por Andar} e
\textit{Tempo de Jornada por Andar}.

\unsure{Será que também botamos um exemplo ou somente uma breve descrição,
  pra deixar esta seção mais simples? (Vanzella)

  Resposta: dizer ``mostraremos exemplos na seção XXX''}

\begin{description}[leftmargin=!,labelwidth=\widthof{\bfseries Tempo de Jornada por Andar}]
  \item[Clientes por Elevador]
    Este gráfico mostra a ocupação total de cada elevador, durante o período de
    simulação, em um formato de barras.
  \item[Chegadas por Andar]
    Este gráfico em barras ilustra a quantidade de clientes que foram gerados pelo
    simulador em cada andar.
  \item[Desembarques por Andar]
    O gráfico de Desembarques por Andar plota, através de barras verticais, a
    quantidade de clientes que foram desembarcados em cada andar.
  \item[Tempo de Jornada por Andar]
    Esta matriz de gráficos mostra, utilizando barras verticais, o tempo médio de
    jornada entre cada par de andares.
\end{description}

Os relatórios e gráficos dão base para análise e proposta de uma estratégia (ou
um conjunto de estratégias) para serem implementados em prédios já existentes.

\section{\label{sec:objectives:planning}Planning}

\unsure{Estou na dúvida se esta seção fica e/ou se ela fica nessa posição.
(Taschetto)

Resposta: e o SIMPLE onde é descrito?

Acho que sai daqui, sim.}

Os algoritmos de \textit{planning} se mostraram os mais promissores para a
realização desta tarefa, dadas as limitações de tempo de desenvolvimento e a
quantidade de dados disponíveis em sistemas reais de elevadores.

Um algoritmo de \textit{planning}, conforme descrito na
seção~\ref{sec:ai:planning}, utiliza uma função de custo (descrita na
seção~\ref{sec:ai:minimize-cost-function}) e tenta encontrar uma sequência de
eventos que gere a menor soma de custos. Estes eventos podem ser quaisquer
eventos que o sistema tenha controle. No caso de sistemas de elevadores, é
possível controlar, por exemplo, qual elevador atenderá qual chamado em uma
fila. Levando-se em conta os dados disponíveis no momento e, talvez, fazendo
inferências a respeito de outros, calcula-se o custo por alguns passos no futuro
(por exemplo, atendendo vários eventos da fila). Um exemplo deste comportamento
pode ser visto na seção~\ref{sec:ai:planning}.

\unsure{Pedro, revisar pois diz que o planning usa função de custos e isso não é mais verdade.}

O algoritmo de \textit{planning} será comparado com os comportamentos mais
triviais, como os descritos nas seções~\ref{sec:ai:nn}~e~\ref{sec:ai:nnm}. Além
disto, a comparação de diferentes funções de custo e diferentes
horizontes\footnote{Horizonte é a quantidade de passos no futuro que o algoritmo
prevê. Mais detalhes podem ser vistos na seção~\ref{sec:ai:planning}.} será
feita, avaliando quão vantajosa é a implementação de um destes algoritmos,
comparado com a implementação dos algoritmos triviais.

\section{\label{section:scenarios}Cenários de testes}

Pode-se dizer que cada prédio é um cenário em potencial no contexto deste
trabalho. Os atributos que podem ser utilizados para definir um cenário são:

\begin{description}[leftmargin=!,labelwidth=\widthof{\bfseries F}]
  \item[F]
  Número total de andares do prédio.
  \item[E]
  Número total de elevadores que compõem o sistema.
  \item[C]
  Capacidade\footnote{Como efeito de redução de escopo, consideram-se todos os
  elevadores de um mesmo sistema como tendo a mesma capacidade.} (em número de
  pessoas\footnote{Considerando uma pessoa com peso médio de 70 kg.}) máxima de
  passageiros que cada elevador é capaz de transportar.
  \item[D]
  Conjunto\footnote{Cada andar do prédio possui uma distribuição distinta.} de distribuições de probabilidade de chegada de passageiros.
\end{description}

Logo, há tantos cenários possíveis quanto há prédios ao redor do mundo. Os
cenários de testes serão limitados em algumas categorias. A escolha destas
divisões segue a classificação \cite{Emporis15} sugerida pela empresa
\textbf{Emporis GmbH}, uma empresa de mineração de dados sobre imóveis com sede
em Frankfurt. O limite inferior de 4 andares é devido à exigência legal do
município de Porto Alegre\footnote{http://www2.portoalegre.rs.gov.br/netahtml
/sirel/atos/Lei\%201344} onde prédios deste tamanho ou maiores (mas não menores
que isto) devem ser construídos com elevadores. O limite superior é de 163
andares\footnote{O maior arranha-céu do mundo em 2015 é o Burj Khalif,
localizado em Dubai, com mais de 800m de altura distribuídos em 163 andares
habitáveis.}.

Os cenários são definidos na tabela \ref{tab:cenarios}. Serão simulados os
tamanhos de prédios limítrofes superiores de cada categoria combinados com as
quantidades de elevadores limítrofes superiores estabelecidas de forma
arbitrária.

\begin{table}[htb!]
\centering
\caption{Categorias de cenários de testes.}
\label{tab:cenarios}
\begin{tabular}{|c|c|c|c|c|c|}
\hline
{\bf Cenário} & {\bf Altura} & {\bf F}  & {\bf E} & {\bf C}
\\ \hline
{\it Low-rise}   & menor que 35 m    & 4 a 11         & 1 a 2   & 6  \\ \hline
{\it High-rise}  & entre 35 e 100 m  & 12 a 39        & 5 a 8   & 10 \\ \hline
{\it Skyscraper} & maior que 100 m   & 40 a 163       & 10 a 16 & 12 \\ \hline
\end{tabular}
\end{table}

\subsection{Parâmetros não utilizados}

Em virtude da limitação de tempo para a elaboração deste trabalho, dois
parâmetros foram removidos do escopo da simulação:

\begin{description}[leftmargin=!,labelwidth=\widthof{\bfseries Pu}]
  \item[P]
  População total do prédio. Com a remoção deste parâmetro considera-se que a
  população do prédio é infinita - ou seja, enquanto a simulação estiver sendo
  executada, novos passageiros continuarão a surgir~-~respeitando a distribuição
  de probabilidade.

  \item[Pu]
  Propósito do prédio: \textit{residencial} (com baixo fluxo entre os andares
  superiores), \textit{comercial com múltiplas empresas} (com médio fluxo entre
  andares superiores) ou \textit{comercial com única empresa} (com alto fluxo
  entre andares superiores)\footnote{O propósito do prédio é diretamente
  relacionado com \textbf{D}.}. Com a remoção deste parâmetro considera-se que
  a probabilidade de um passageiro chegar e ir para qualquer andar do prédio é
  sempre a mesma.
\end{description}

Acredita-se que a simplificação resultante destas remoções não vá impactar os
resultados obtidos de forma negativa.
