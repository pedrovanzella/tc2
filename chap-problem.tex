\chapter{\label{chap:problem}Descrição do Problema}

A maior parte dos prédios comerciais possui instalações de grupos com 2 a 8
elevadores \cite{KOEHLEROTTIGER02}. Em muitos prédios, esses são o meio de
transporte primário entre andares, visto que escadas são menos práticas e,
muitas vezes, menos acessíveis ou exclusivas para situações de emergência. Dado
o seu uso em larga escala, a ineficiência dos sistemas de controle de elevadores
é percebida diariamente por seus usuários sob a forma de tempo despendido. Neste
cenário, deseja-se encontrar formas de otimizar estes sistemas de controle e
refletir positivamente no desempenho geral do sistema e na percepção da
qualidade por seus usuários.

O problema estudado neste trabalho é modelado da seguinte forma: seja um prédio
com \textbf{F} andares e \textbf{E} elevadores, cada um com capacidade para
transportar \textbf{C} pessoas simultaneamente; sabe-se que a chegada de
passageiros, em cada andar, obedece uma função de distribuição de probabilidade
\textbf{D}, distinta das demais. De que forma o sistema pode atender os
passageiros de modo a minimizar o tempo médio de atendimento das pessoas em um
dado intervalo de tempo?

\section{\label{section:egcs}Sistema de controle de grupo de elevadores}

Um \textit{Elevator Group Control System} (\textbf{EGCS}), ou sistema de
controle de grupo de elevadores, é responsável por coordenar as ações dos
elevadores do prédio~\cite{kuzunuki1984elevator}. Esta coordenação visa atender
a todas as \textbf{chamadas de corredor}\footnote{Passageiros estão fora dos
elevadores e realizam uma chamada para subir ou descer a partir do andar em que
se encontram.} e \textbf{chamadas de cabine}\footnote{Passageiros estão dentro
do elevador e realizam uma chamada para desembarcar em um andar destino.} em um
dado instante. Para isto, utiliza como entrada um conjunto de dados que modelam
o \textbf{estado atual do sistema}. Este conjunto de dados pode ser modelado,
minimamente, pelas seguintes informações:\unsure{Devemos mostrar o ``mínimo'' aqui
e na seção da modelagem apresentar como nós fizemos neste estudo ou já apresentar
aqui a nossa modelagem? (Taschetto)}

\begin{itemize}
  \item Para cada andar:
  \begin{itemize}
    \item Se existe uma \textbf{chamada de corredor} a partir deste andar e o
          sentido (subir, descer ou ambos);
  \end{itemize}
  \item Para cada elevador:
  \begin{itemize}
    \item Um conjunto de \textbf{chamadas de cabine} solicitadas pelos
          passageiros embarcados;
    \item O andar em que se encontra;
    \item Se está parado, subindo ou descendo;
    \item Sua lotação atual\footnote{A capacidade máxima \textbf{C} é uma
          propriedade do problema e é igual para todos os elevadores que compõem
          o sistema.}.
  \end{itemize}
\end{itemize}

Com base nestes dados o EGCS utiliza-se de algoritmos e técnicas para fornecer
como saída:

\begin{itemize}
  \item Para uma nova \textbf{chamada de corredor}:
  \begin{itemize}
    \item Qual elevador deverá atender esta chamada de modo a minimizar o tempo de espera.
  \end{itemize}
\end{itemize}

Para cada ocorrência deste evento devem existir estampas de tempo e informações
adicionais associadas\footnote{Sobre o andar, o elevador e os passageiros.} para
fins estatísticos e cálculos de métricas, conforme apresentado a seguir.

\section{\label{section:data}Aquisição de dados e métricas}

Embora a aquisição dos dados sobre o estado atual do sistema\footnote{Conforme
visto na seção \ref{section:egcs}.} seja trivial para a tecnologia atual,
através dos sensores e atuadores distribuídos ao longo da instalação do sistema,
obter dados mais complexos sobre o tráfego de um sistema de elevadores ainda é
um problema em aberto na indústria~\cite{KOEHLEROTTIGER02}. Em casos simples é
possível designar pessoas para observar e contar passageiros entrando e saindo
dos elevadores. Já em casos mais complexos foram aplicadas soluções mais
engenhosas, como contagem de pessoas aplicando algoritmos de visão computacional
em câmeras de segurança e sensores de carga. Estas abordagens possuem fatores
complicadores~-~como, por exemplo, a dificuldade em lidar com as diferenças de
iluminação, baixa qualidade de vídeo, baixa precisão de sensores, etc - e o
resultado obtido não compensa o custo~\cite{KOEHLEROTTIGER02}.

Neste contexto, a simulação de sistemas de elevadores torna-se uma alternativa
atraente para obter métricas e avaliar o desempenho de um sistema na fase de
projeto, antes da sua construção. Por exemplo, supondo que em um prédio de 10
andares com 2 elevadores exista uma fila de 20 pessoas que desejam ir a andares
variados do prédio; cada uma destas pessoas chegou à fila em um momento
distinto; porém, a chamada de corredor foi realizada apenas pela primeira pessoa
da fila. Como medir o tempo que cada passageiro ficou esperando pelo elevador
chegar, sendo que somente a primeira pessoa da fila interagiu diretamente com o
sistema? Ou como medir o tempo que cada pessoa levou para chegar ao seu destino?
A modelagem do problema e simulação em ambiente computacional permitem obter
dados que em um ambiente real seriam impossíveis ou muito caros de se conseguir.
Tais dados podem compreender:

\begin{itemize}
  \item Para cada andar:
  \begin{itemize}
    \item Uma fila dos passageiros que encontram-se esperando naquele andar e querem subir;
    \item Uma fila dos passageiros que encontram-se esperando naquele andar e querem descer;
  \end{itemize}
  \item Para cada passageiro:
  \begin{itemize}
    \item A hora de sua chegada na fila;
    \item O seu andar de origem;
    \item O seu andar de destino;
    \item A hora de embarque no elevador;
    \item A hora de desembarque no andar de destino;
  \end{itemize}
  \item Para cada elevador:
  \begin{itemize}
    \item O conjunto dos passageiros que encontram-se dentro daquele elevador e suas informações associadas (vide item anterior).
  \end{itemize}
\end{itemize}

A partir destes dados, algumas das principais métricas utilizadas pela indústria
de elevadores para avaliar o desempenho de seus sistemas podem ser calculadas:

\begin{description}[leftmargin=!,labelwidth=\widthof{\bfseries HC\textsubscript{5\%}}]
  \item[HC\textsubscript{5\%}]
  Percentual da população total do prédio que um sistema de elevadores consegue
  transportar em um intervalo de 5 minutos. Um HC\textsubscript{5\%} aceitável é
   de no mínimo 14\%~\cite{KOEHLEROTTIGER02}. Por exemplo, em um prédio cuja
   população é de 600 pessoas, este índice representa o transporte de no mínimo
   84 pessoas em 5 minutos.\unsure{Precisamos falar disso uma vez que não
   simulamos este tipo de situação? (Taschetto)}

  \item[WT]
  \textit{Waiting Time}, ou tempo de espera; está compreendido entre a
  requisição por passageiro, ou sua chegada na fila, e o seu embarque em um
  elevador.

  \item[JT]
  \textit{Journey Time}, ou tempo de jornada; está compreendido entre o embarque
   de um passageiro em um elevador e o posterior desembarque em seu destino.

  \item[ST]
  \textit{System Time}, ou tempo de sistema; está compreendido entre entre a
  chegada de um passageiro e o desembarque em seu destino, ou seja, é a soma do
  tempo de espera com o tempo de jornada.

  \item[AWT]
  \textit{Average Waiting Time}, ou tempo médio de espera.

  \item[AJT]
  \textit{Average Journey Time}, ou tempo médio de jornada.

  \item[AST]
  \textit{Average System Time}, ou tempo médio de sistema.

  \item[RTT]
  \textit{Round-trip Time}, ou tempo de ida e volta em uma tradução livre; é o
  tempo médio de uma viagem de um elevador partindo do lobby, indo até todos os
  andares do prédio e de volta ao lobby em horário de pico.\unsure{Precisamos
  falar disso uma vez que não simulamos este tipo de situação? (Taschetto)}
\end{description}

O tempo médio de sistema (AST) define a qualidade do serviço, já que está ligado
diretamente à percepção que os passageiros possuem do sistema. É correto afirmar
que o desejo de um passageiro é chegar no seu destino o mais rápido
possível~-~ou seja, com o menor tempo de sistema possível. Normalmente, tempos
de sistema menores relacionam-se com um alto HC\textsubscript{5\%}; porém,
passageiros tendem a dar maior importância a um baixo tempo de espera (WT) do
que a um baixo tempo de jornada (JT) ~\cite{KOEHLEROTTIGER02}. Isto por que, uma
vez dentro do elevador, o passageiro não se sente mais esperando: ele sente que
já está sendo servido. Ainda assim, embora uma redução de 32 para 28 segundos de
espera não seja considerada uma grande melhoria, é psicologicamente importante
evitar esperas longas, i.e.,~60 segundos ou mais.

\section{\label{section:difficulties}Situações fora do escopo deste trabalho}

Existem fatores humanos cujas ocorrências geram situações especiais que aumentam
a complexidade do problema e podem prejudicar o desempenho geral do sistema. Por
exemplo, podem ser destacadas as seguintes situações:

\begin{itemize}
  \item Um passageiro segura a porta do elevador aberta para alguém atrasado~-~
        deste modo, o passageiro que segura a porta está sendo altruísta com o
        atrasado, porém egoísta em relação aos outros passageiros que estão
        aguardando nos demais andares;
  \item Um passageiros acidentalmente faz uma chamada de corredor no sentido
        errado;
  \item Um passageiro faz uma chamada de corredor, desiste de aguardar pelo
        elevador e utiliza as escadas - ou seja, não embarca no elevador;
  \item Um passageiro acidentalmente faz uma chamada de cabine para o andar
        errado;
\end{itemize}

Estes são apenas alguns exemplos de situações que ocorrem diariamente. Alguns
destes problemas podem ser mitigados através de aprimoramentos da interface dos
elevadores. Por exemplo, seleções acidentais de andares ou desistências poderiam
ser desfeitas, caso houvesse uma interface para cancelar chamadas de corredor ou
de cabine.

No entanto, propor este tipo de análise e alteração de interfaces e
comportamentos não está no escopo deste estudo. O objetivo aqui é analisar e
propor alterações somente nos sistemas de controle que regem o comportamento dos
elevadores da maneira com que estão atualmente instalados na vasta maioria dos
prédios.