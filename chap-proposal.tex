\chapter{\label{chap:proposal}Proposta do Trabalho}

A proposta deste trabalho é comparar, através de simulações, diferentes
estratégias de controle de elevadores utilizando Inteligência Artificial em
cenários distintos. Os resultados das simulações serão avaliados e, dentre as
opções possíveis, serão verificadas quais estratégias resultam em um melhor
desempenho no transporte de passageiros para cada cenário. O objetivo é que esta
solução reduza significamente o tempo que passageiros dependem de elevadores
(seja aguardando ou em viagem) e possa ser implementado em grupos de elevadores
já existentes.

Para realizar tais simulações, será modelado e desenvolvido um simulador de
elevadores. Esta ferramenta irá expor uma \textit{Application Programming
Interface} (API) com elementos que permitam ao usuário definir um cenário
utilizando os parâmetros\footnote{Parâmetros descritos na
seção~\ref{section:scenarios}. A ausência dos parâmetros \textbf{P} (população)
e \textbf{Pu} (propósito) se dá em função da limitação de tempo para a
realização deste estudo. A população será considerada infinita e o propósito
será considerado ``comercial com múltiplas empresas''. Acredita-se que esta
simplificação será boa o suficiente para considerar os resultados válidos.}
\textbf{F} (andares), \textbf{E} (elevadores), \textbf{C} (capacidade dos
elevadores) e \textbf{D} (distribuição de probabilidade de chegada de
passageiros). Além disto, a API permitirá que o usuário selecione a estratégia a
ser utilizada pelo sistema de controle de grupos de elevadores.
Alternativamente, o usuário poderá utilizar suas próprias implementações.

O simulador deverá fornecer como saída um relatório apresentando
métricas\footnote{Seção~\ref{section:data}.} de desempenho para sistemas de
controle de grupo de elevadores. Será possível simular uma estratégia em um
conjunto de cenários ou comparar várias estratégias em um mesmo cenário. Estes
dados darão base para análise e proposta de uma estratégia (ou um conjunto de
estratégias) que pode ser imediatamente implementada em um prédio existente.

Juntamente com o simulador de elevadores, pretende-se implementar, pelo menos,
duas diferentes estratégias de Inteligência Artificial para realização da
análise de desempenho nos cenários definidos
anteriormente\footnote{Seção~\ref{section:scenarios}.}, de modo a determinar se
há um ganho significativo em se utilizar uma delas em prédios de baixo, médio e
grande porte, tanto residenciais quanto comerciais. As estratégias pretendidas
são detalhadas a seguir.

\section{\label{sec:proposal:planning}Planning}

Os algoritmos de \textit{planning} se mostraram os mais promissores para a
realização desta tarefa, dadas as limitações de tempo de desenvolvimento e a
quantidade de dados disponíveis em sistemas reais de elevadores.

Um algoritmo de \textit{planning}, conforme descrito na
seção~\ref{sec:ai:planning}, utiliza uma função de custo (descrito na
seção~\ref{sec:ai:minimize-cost-function}) e tenta encontrar uma seqüência de
eventos que gere a menor soma de custos. Estes eventos podem ser quaisquer
eventos que o sistema tenha controle. No caso de sistemas de elevadores, é
possível controlar, por exemplo, qual elevador atenderá qual chamado em uma
fila. Levando-se em conta os dados disponíveis no momento e, talvez, fazendo
inferências a respeito de outros, calcula-se o custo por alguns passos no futuro
(por exemplo, atendendo vários eventos da fila). Um exemplo claro deste
comportamento pode ser visto na seção~\ref{sec:ai:planning}.

O algoritmo de \textit{planning} será comparado com os comportamentos mais
triviais, como os descritos nas seções~\ref{sec:ai:nn}~e~\ref{sec:ai:nnm}. Além
disto, a comparação de diferentes funções de custo e diferentes
horizontes\footnote{Horizonte é a quantidade de passos no futuro que o algoritmo
prevê. Mais detalhes podem ser vistos na seção~\ref{sec:ai:planning}.} será
feita, avaliando quão vantajosa é a implementação de um destes algoritmos,
comparado com a implementação dos algoritmos triviais.
