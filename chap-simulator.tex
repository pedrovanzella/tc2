\chapter{\label{chap:simulator}Simulador}

\unsure{Capítulo sendo alterado por Taschetto.}

De acordo com Banks~(2005, p.~3):

\begin{directcite}
Simulação é a imitação da operação de um sistema ou processo do mundo real ao
longo do tempo. Ela envolve a geração de uma história artificial de um sistema
ou processo e a observação desta história de modo a realizar inferências a
respeito das características operacionais da realidade ali representada.
\end{directcite}

Entretanto, a simulação não é a única abordagem existente para se estudar e
compreender um sistema e suas características. É senso comum que cada sistema,
possuindo suas próprias características e idiossincrasias, deve ser analisado
através da ferramenta correta. Logo, embora a simulação pareça ser uma boa
alternativa à primeira vista em diversos casos, é possível que ela não seja a
forma mais apropriada para o seu estudo. Assim, se faz importante a existência
de métodos objetivos para que se possa verificar se a simulação é realmente a
ferramenta apropriada para cada caso estudado.

\section{\label{simulator:motivation}Motivação}

A fim de decidir de forma objetiva qual a melhor abordagem para um dado sistema,
Law~\cite{Law} propõe uma reflexão (figura~\ref{fig:systemstudy}) através das
seguintes perguntas:

\begin{figure}[htb!]
\centering\includegraphics{img/systemstudy.eps}
\caption[Formas de estudar um sistema]{\label{fig:systemstudy}Formas de estudar um sistema. Fonte:~\cite{Law}}
\end{figure}

\begin{enumerate}
\item \textit{Experimentar com o próprio sistema ou experimentar com um modelo
do mesmo?}

Um sistema de elevadores operacional e pronto para realizar as experimentações
não encontra-se entre os recursos disponíveis para a elaboração da pesquisa.
Ainda, mesmo que tal estrutura estivesse disponível, os cenários de testes
possíveis seriam limitados pelas restrições físicas do sistema, tornando o
estudo das melhorias mais caro e menos flexível. Logo, optou-se pela utilização
de um \textit{modelo do sistema}.

\item \textit{Experimentar utilizando um modelo físico do sistema ou utilizar um
modelo matemático?}

Um modelo físico de um sistema de elevadores poderia ser constituído por uma
maquete de um prédio com mini-elevadores movidos à motores de passo, que por
sua vez seriam controlados por microcontroladores programáveis conectados à uma
rede de sensores. Este projeto por si só, entretanto, já seria grandioso
demais~-~além de, obviamente, fugir do escopo da Ciência da Computação e ser
mais adequado à um trabalho de conclusão de Engenharia Elétrica ou Engenharia de
Controle e Automação. Além disso, da mesma forma que em um sistema real de
elevadores, os cenários de testes possíveis seriam limitados pelas restrições
físicas do modelo do sistema. Portanto, optou-se pela utilização de um
\textit{modelo matemático do sistema}, reproduzível em ambiente computacional e
parametrizável para diferentes cenários.

\item \textit{O problema pode ser resolvido de forma analítica?}

Conforme afirmado no capítulo \ref{chap:intro} deste estudo, o problema de
atribuir elevadores para atender chamadas feitas pelos passageiros minimizando
alguma métrica encontra-se no conjunto de problemas NP-difícil (ou
NP-hard, ou NP-complexo)~\cite{SeKo99}. Assim, uma solução ótima, computável
em tempo polinomial, ainda não é conhecida para este problema. Este fato vai ao
encontro dos grandes esforços da indústria em procurar soluções para resolver o
problema ao longo das décadas, não poupando esforços e investimentos em busca
desta solução.

\end{enumerate}

Portanto, se justifica a escolha da simulação de um modelo matemático do sistema
como uma forma apropriada para o estudo e experimentação de um sistema de
elevadores.

\subsection{\label{simulator:motivation:classification}Classificação do modelo de simulação}

A partir de um modelo matemático a ser estudado por meio de simulação (doravante
chamado de \textit{modelo de simulação}), o mesmo pode ser classificado em três
dimensões~\cite{Banks,Law}:

\begin{enumerate}
\item \textit{Estático ou Dinâmico}

O modelo de simulação neste escopo é dinâmico. Deseja-se verificar o
comportamento do sistema ao longo de um intervalo de tempo finito, à medida que
passageiros chegam e elevadores os transportam através dos andares do prédio.

\item \textit{Determinístico ou Estocástico}

Um modelo de simulação que não possua nenhum componente probabilístico (i.~e,~
aleatoriedade) é chamado de \textit{determinístico}. Neste tipo de modelo, a
saída fornecida pelo modelo de simulação será sempre a mesma para uma mesma
entrada. Porém a grande maioria dos sistemas do mundo real possuem, no mínimo,
algum grau de aleatoridade na sua entrada - por isso são chamados
\textit{estocásticos}. Estes, diferentemente de modelos
\textit{determinísticos}, fornecem uma saída igualmente aleatória~-~e, por esta
razão, esta saída deve ser considerada como um conjunto de \textit{estimativas}
das características reais do sistema, e não as características propriamente
ditas \cite{Banks}.

Em um \textbf{EGCS} real não é possível prever quantos passageiros utilizarão o
sistema, quando eles chegarão e tampouco para qual andar irão. Portanto, um
modelo de simulação válido neste escopo deve ser \textit{estocástico} de modo a
lidar com a aleatoridade da entrada de passageiros no sistema.

\unsure{Fiquei na dúvida. Um sistema que utilize geradores pseudo-aleatórios com
sementes pode ser considerado REALMENTE estocástico? (Taschetto)}

\item \textit{Contínuo ou Discreto}

A figura \ref{fig:disccont} ilustra o comportamento de uma variável de estado em
modelos de simulação \textit{contínuos} e \textit{discretos}. O modelo
\textit{contínuo} é aquele onde os valores das variáveis mudam continuamente ao
longo do tempo. Já em um modelo \textit{discreto} as variáveis de estado tem seu
valor alterado em instantes separados do tempo \cite{Banks}. Para este projeto,
não há a necessidade informações instantâneas a respeito da movimentação de
passageiros e elevadores, e sim de observar o comportamento do sistema dada a
ocorrência de determinados eventos. Portanto, o modelo de simulação utilizado
neste estudo é \textit{discreto}.

\begin{figure}[htb!]
\centering\includegraphics{img/discrete_continuous.eps}
\caption[Variável de estado em um modelo contínuo e discreto]{\label{fig:disccont}Variável de estado em um modelo contínuo (A) e discreto (B). Fonte:~\cite{Banks}}
\end{figure}

\end{enumerate}

\subsection{\label{simulator:movation:discrete}Simulação baseada em eventos discretos}

Segundo Law~(2000, p.~6):

\begin{directcite}
A Simulação de Eventos Discretos (\textit{Discrete-Event Simulation}) compreende
a modelagem de um sistema à medida que ele \textbf{evolui ao longo do tempo}
através de uma representação na qual as variáveis de estado são alteradas
instantaneamente em \textbf{instantes separados no tempo}.
\end{directcite}

A abordagem sugerida por Law é o padrão de projeto de simuladores ao utilizar-se
um modelo de simulação \textit{dinâmico}, \textit{estocástico} e
\textit{discreto} para representar o sistema do mundo real. Em função da
natureza dinâmica desta abordagem, é necessário acompanhar o valor atual do
tempo da simulação à medida que a simulação é executada, armazenando este valor
em uma variável. Esta variável é chamada de \textit{relógio da simulação}
\cite{Law}. Geralmente, não há relação entre o tempo de simulação e o tempo
necessário para a simulação ser executada. Por exemplo, um experimento pode
simular o funcionamento de um banco entre 9h e 17h (tempo de simulação), mas o
tempo necessário para executar a simulação poderia ser de 4 minutos.

Também se faz necessário um mecanismo de avanço de tempo que gerencie o valor
desta variável. Neste simulador é utilizada a abordagem de \textit{avanço de
tempo para o próximo evento}, onde o \textit{relógio da simulação} é
inicializado em 0 e é determinado em que ponto tempo ocorrerão eventos
futuros~-~em outras palavras, é feito o agendamento de eventos. Então, o
\textit{relógio da simulação} é avançado para o tempo da ocorrência do
\textit{primeiro} destes eventos futuros. Neste ponto do tempo, o estado do
modelo é atualizado de acordo com o evento que ocorreu e novas ocorrências de
eventos futuros são agendadas. Então, o \textit{relógio da simulação} avança
para o instante da ocorrência do \textit{novo} primeiro dos eventos futuros e o
estado do sistema é atualizado em função da ocorrência deste evento. Este
processo repete-se até que uma condição de parada previamente definida seja
satisfeita.

Uma vez que toda alteração no estado do sistema ocorre na ocasião de um evento,
os períodos de espera, que são o tempo entre a ocorrência de um evento e o
próximo~-~não são relevantes para a simulação. Afinal, o estado do sistema não
foi alterado neste ínterim~-~ou seja, nada de interessante ocorreu durante
aquele tempo \cite{Sim}. Deste modo, é possível reduzir o esforço computacional
necessário para executar a simulação.

Um exemplo desta dinâmica é ilustrado pela figura \ref{fig:nextevent}. O eixo do
tempo, iniciado em 0, marca os tempos agendados para os eventos $\{e_{0}, e_{1},
e_{2}, e_{3}, e_{4}, e_{5}\}$. As setas indicam os valores assumidos pelo
\textit{relógio da simulação}, ou seja, $\{t_{0}, t_{1}, t_{2}, t_{3}, t_{4},
t_{5}\}$. Observa-se que $t_{0} = e_{0}$, $t_{1} = e_{1}$ e assim
sucessivamente~-~ou seja, o \textit{relógio da simulação} avança diretamente
para o momento da ocorrência do próximo evento.

\begin{figure}[htb!]
\centering\includegraphics{img/nextevent.eps}
\caption[Avanço de tempo para o próximo evento]{\label{fig:nextevent}Ilustração da evolução do \textit{relógio da simulação} utilizando a abordagem de \textit{avanço de tempo para o próximo evento}. Fonte:~\cite{Law}}
\end{figure}

Sendo assim, o modelo deste estudo será uma simulação de eventos discretos com avanço de tempo para o próximo evento.

\section{\label{simulator:requirements}Requisitos de Projeto}

O projeto foi concebido de forma a garantir alguns requisitos considerados de
suma importância em relação ao simulador e as simulações. São elas:

\begin{description}[leftmargin=!,labelwidth=\widthof{\bfseries Determinístico}]

  \item[Configurável]

  Deve ser possível definir cenários de forma simples e sem a necessidade de
  recompilar ou reconstruir os binários do simulador.

  Para que isto fosse possível, os cenários para simulação são armazenados em um
  arquivo de configuração no formato \textit{YAML}\footnote{\textit{YAML Ain't
  Markup Language}, http://yaml.org/} e são carregados pelo simulador em tempo
  de execução com o auxílio da biblioteca \texttt{yaml-cpp}\footnote{\textit
  {yaml-cpp is a YAML parser and emitter in C++}, https://github.com/jbeder
  /yaml-cpp}. O arquivo de configuração é detalhado na seção
  \ref{simulator:model:config}.

  \item[Determinístico]

  Deve ser possível reproduzir simulações de maneira idêntica, independente do
  horário e da ordem em que for iniciada. Isto é especialmente importante ao
  comparar o mesmo cenário sob diferentes estratégias, quando é imperativo
  garantir que os passageiros cheguem nos mesmos instantes e locais em todas as
  simulações de um mesmo cenário.

  Para que isto fosse possível, os cenários configurados no arquivo de
  configuração contém, dentre outras informações, uma semente textual utilizada
  para inicializar os geradores de números aleatórios e distribuições de
  probabilidade dentro do simulador.

  \item[Escalável]

  Deve ser possível simular diversos cenários em uma mesma execução do simulador
  e em um tempo aceitável: alguns segundos para cenários simples e até 30
  minutos para cenários complexos.

  Para que isto fosse possível, optou-se por desenvolver o simulador na
  linguagem \texttt{C++11}. Tal linguagem é tida na indústria de software como
  capaz de oferecer desempenho superior às demais linguagens de propósito
  geral~-~devido, em grande parte, ao fato de ser compilada em linguagem de
  máquina e permitir ao programador gerenciar diretamente o uso de memória do
  processo. Além disso, juntamente de sua biblioteca padrão, o \texttt{C++11}
  oferece um conjunto atraente de estruturas de dados, ferramentas de
  desenvolvimento e depuração.

  \item[Rastreável]

  Para cada simulação, o simulador deve gerar arquivos de \textit{log},
  detalhando os eventos e situações ocorridos durante a simulação no escopo
  daquele cenário.

  Para que isto fosse possível, foi utilizada a biblioteca
  \texttt{glog}\footnote{\textit{C++ implementation of the Google logging
  module}, https://github.com/google/glog}, desenvolvido pelo \textit{Google} e
  com implementações para várias linguagens, dentre elas o \texttt{C++}.

  \item[Testável]

  Deve ser possível desenvolver e executar testes unitários nos componentes do
  simulador.

  Para que isto fosse possível, foi utilizada a biblioteca \texttt{Google
  Test}\footnote{\textit{Google's C++ test framework},
  https://github.com/google/googletest}, desenvolvido pelo \textit{Google} e
  com implementações para várias linguagens, dentre elas o \texttt{C++}.

\end{description}

\section{\label{simulator:model}Modelo do Simulador}

\subsection{\label{simulator:model:scenario}Cenário}

Como entrada, o simulador recebe um conjunto de cenários, definidos no arquivo
de configuração (seção \ref{simulator:model:config}) e realiza a simulação dos
mesmos. Cada cenário é composto pelas seguintes informações:

\begin{description}[leftmargin=!,labelwidth=\widthof{\bfseries Função de Custo}]
  \item[Nome]

  Identificação textual do cenário.

  \item[Duração]

  Tempo durante o qual serão geradas chegadas de clientes aos andares do prédio.
  Após este tempo, não chegarão mais clientes. Porém, clientes que já entraram
  no prédios e estão localizados nos andares ou nos elevadores precisam terminar
  suas viagens antes que a simulação chegue ao fim. Devido a isto, é normal que
  o tempo real de simulação seja maior que a duração estipulada no cenário.

  \item[Agendamento]

  Lista de algoritmos de agendamento.

  \item[Horizonte]

  Horizonte de expansão do agendamento com algoritmo de \textit{planning} (não utilizado no agendamento \textit{simple}).

  \item[Função de Custo]

  Lista de funções de custo.

  \item[Semente]

  Semente textual para inicialização os geradores de números aleatórias.

  \item[Elevadores]

  Número de elevatores do prédio.

  \item[Capacidade]

  Capacidade dos elevadores (a mesma para cada elevador).

  \item[Andares]

  Lista com o intervalo médio\footnote{O valor médio informado é utilizado como
  parâmetro $\lambda$ de entrada para uma Distribuição de Poisson. Cada andar do
  prédio possui uma distribuição distinta.} de chegada de passageiros, em
  segundos, para cada andar. A lista começa com a média do andar térreo e segue
  sucessivamente. O número de andares é igual ao tamanho da lista.

\end{description}

Convém salientar que, definido um cenário, será realizada uma simulação para cada
combinação possível de algoritmo de agendamento com função de custo. Por
exemplo, se forem especificados os algoritmos de agendamento \textit{simple} e
\textit{planning} juntamente das funções de custo \textit{nearest neighbour} e
\textit{weighted}, serão realizadas 4 simulações deste cenário com as seguintes
combinações:

\begin{enumerate}
  \item Agendamento \textit{simple} com função de custo \textit{nearest neighbour};
  \item Agendamento \textit{simple} com função de custo \textit{weighted};
  \item Agendamento \textit{planning} com função de custo \textit{nearest neighbour};
  \item Agendamento \textit{planning} com função de custo \textit{weighted}.
\end{enumerate}

\subsection{\label{simulator:model:config}Arquivo de Configuração}

A entrada de dados para o simulator se dá através de um arquivo de configuração
chamado \texttt{config.yaml}. Este arquivo obedece o padrão \textit{YAML}, um
formato de serialização de dados legíveis por humanos.

\begin{algorithm}[htb]
  \centering
    \begin{minted}[frame=lines,framesep=2mm,linenos,fontsize=\small]{yaml}
scenarios:
  - name: Scenario 1
    duration: 43200 # 12 hours
    scheduler: [ 0, 1 ] # simple, planning
    planningHorizon: 5
    cost_function: [ 1, 3, 4 ] # randon, bnn, weighted
    seed: 54TH7hboAG1iOsDIDhJp
    elevators: 2
    capacity: 6
    floors: [ 60, 520, 360, 240, 240, 90, 90, 90 ]

  - name: Scenario 2
    duration: 86400 # 24 hours
    scheduler: [ 1 ] # planning
    planningHorizon: 3
    cost_function: [ 1, 2, 3, 4 ] # random, nn, bnn, weighted
    seed: w9JwgykwejtoL2icSgHo
    elevators: 6
    capacity: 10
    floors: [ 60, 520, 520, 520, 360, 360, 360, 360,
             360, 360, 240, 240, 240, 240, 240, 240 ]
    \end{minted}
  \caption{Arquivo de configuração definindo dois cenários distintos.}
  \label{alg:config}
\end{algorithm}

No exemplo de arquivo de configuração ilustrado no algoritmo \ref{alg:config},
são definidos dois cenários. O primeiro, chamado de \texttt{Scenario 1}, onde
clientes chegarão durante 12 horas distribuídos ao longo dos 8 andares do prédio
e serão atendidos por um dos 2 elevadores, cuja capacidade é de 8 passageiros
cada. Já no cenário \texttt{Scenario 2} passageiros chegarão num período de 24
horas, distribuídos ao longo dos 16 andares e serão atendidos por 6 elevadores,
com capacidade  de transportar 10 pessoas cada.

\subsection{\label{simulator:model:clock}Relógio}
\lipsum[1]

\subsection{\label{simulator:model:events}Eventos}
\lipsum[1]

\subsection{\label{simulator:model:notification}Notificação de Eventos}
\lipsum[1]

\subsection{\label{simulator:model:queue}Fila de Eventos}
\lipsum[1]

\subsection{\label{simulator:model:statistics}Estatísticas}
\lipsum[1]

\subsection{\label{simulator:model:simulator}Simulador}
\lipsum[1]
